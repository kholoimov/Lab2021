\section{Контрольні запитання}
\subsection{Що таке чотириполюсник? У чому полягає відмінність лінійного чотириполюсника від нелінійного? Активного від пасивного?}
\indent \textbf{Чотириполюсник} - це електричне коло (ділянка електричного кола) з чотирма
полюсами, зажимами, клемами або іншими засобами приєднання до нього інших електричних кіл чи ділянок електричних кіл. \\~\\
\textbf{Пасивний чотириполюсник} - це такий чотириполюсник, який не здатний збільшувати потужність вхідного сигналу за рахунок додавання енергії від якогось іншого джерела енергії (внутрішнього чи зовнішнього по відношенню до чотириполюсника). Потужність, що виділяється в елементі кола, підключеного до виходу такого чотириполюсника, менша за потужність, що споживається від джерела сигналу, підключеного до входу чотириполюсника.\\~\\
\textbf{Активний чотириполюсник} - дозволяє збільшувати потужність вихідного сигналу порівняно з потужністю вхідного сигналу за рахунок внутрішніх або зовнішніх
джерел енергії. Має містити активний елемент.\\~\\
\textbf{Лінійний чотириполюсник} - це такий, для якого залежність між струмами, що течуть крізь нього, та напругами на його зажимах є лінійною. Такі чотириполюсники
складаються з лінійних елементів. \\~\\
\textbf{Лінійні елементи електричних кіл} - це такі елементи, параметри яких не залежать від величини струму, що протікає через них або від прикладеної до них напруги.
На виході лінійних чотириполюсників, на відміну від нелінійних, не можуть утворюватися гармоніки ( і т. д.) сигналу частоти , який подано на вхід.\\~\\
\textbf{Нелінійний чотириполюсник} - це такий, який містить нелінійні елементи.
Для нього згадані залежності між струмами та напругами при деяких їх величинах
перестають бути лінійними, а на виході можуть з’являтися гармоніки частот вхідних
сигналів\\~\\
\textbf{Пасивний фільтр} - це пасивний чотириполюсник, який містить реактивні
елементи (індуктивності, ємності), спад напруги на яких або струм через які залежить
від частоти, і завдяки цьому здатен перетворювати спектр сигналу, поданого на його вхід, шляхом послаблення певних спектральних складових вхідного сигналу. Решта
спектральних складових вхідного сигналу проходить через такий пасивний лінійний чотириполюсник, тобто він працює як фільтр для певних спектральних складових сигналу.
Фільтри, побудовані на конденсаторах і резисторах, називють RC-фільтрами.
\subsection{Які пасивні чотириполюсники називаються фільтрами елeктричних сигналів? Що таке АЧХ і ФЧХ фільтрів?}
\textbf{Фільтри електричних сигналів} - пасивні лінійні чотириполюсники, призначенідля виділення певних спектральних складових електричних сигналів. \\
\textbf{АЧХ} - (амплітудно-частотна характеристика)— залежність відношеннямодулів амплітуд вихідного і вхідного гармонічних сигналів від їх частоти, яка є нечим іншим як залежністю модуля коефіцієнта передачі від частоти $\omega$ \\
\textbf{ФЧХ} - (фазо-частотна характеристика) — залежність аргумента комплексного коефіцієнта передачі від частоти, тобто різниці фаз між вихідним і вхідним гармонічними сигналами на частоті $\omega$.

\section{Використані джерела}

\qquad Методичні вказівки до практикуму «Основи радіоелектроніки»
для студентів фізичного факультету / Упоряд. О.В.Слободянюк,
Ю.О.Мягченко, В.М.Кравченко.- К.: Поліграфічний центр «Принт
лайн», 2007.- 120 с.

\qquad Ю.О. Мягченко , Ю. М . Дулич , А.В.Хачатрян “Вивчення
радіоелектронних схем методом комп’ютерного моделювання” :
Методичне видання. – К.: 2006.- с.
