

\qquad \textbf{Діод}  Електроний прилад, що пропускає струм лише в одном напрямку. Має відміну від провідника у власній будові, тому ВАХ діода має певну відмінність від ВАХ резистора. \par
\textbf{Осцилограф} Прилад, що призначений для вимірювання, спостерігання та запису параметрів електричного сигналу. У роботі використовуємо осцилограф для побудови залежності напргуи на діоді від сили струму на ділянці кола. \par
\textbf{Резистор}  пасивний елемент електричного кола, призначений для використання його електричного опору. Основною характеристикою резистора є величина його електричного опору. Для випадку лінійної характеристики, значення електричного струму крізь резистор в залежності від електричної напруги, описується законом Ома. \par
\textbf{Вольт-амперна характеристика}  це залежність сили струму $I_d$ через $p-n$–перехід діода від величини і полярності прикладеної до діода напруги $U_d$.
