\section{Вступна частина}
\setlength{\parindent}{4em}
\indent \textbf{Мета роботи:} За допомогою спектроскопа-гоніометра визначити постійну дифракційної ґратки,
її кутову дисперсію та роздільну здатність, довжини світлових хвиль у спектрі ртутної лампи \par
\textbf{Прилади:} Спектроскоп-гоніометр, плоско паралельна скляна пластинка, дифракційна ґратка,
ртутна лампа
\begin{center}
\textbf{\emph{Теоретичні відомості}}
\end{center}
\qquad Оптична дифракційна ґратка – це скляна або металева пластинка, на якій за допомогою подільної
машини нанесено ряд паралельних штрихів. В учбовій практиці звичайно застосовують так звані
репліки. Найчастіше це желатинові зліпки-копії з металічних відбиваючих ґраток. Ці репліки
розміщують між двома скляними плоско паралельними пластинками. \\
Основними параметрами дифракційної ґратки є її період $d ( d = a + b),$ або постійна ґратки та
число штрихів $N$. \\
Дифракційна ґратка застосовується при такому розміщенні, коли має місце дифракція
Фраунгофера, тобто коли на ґратку падає плоска хвиля, а точка спостереження знаходиться
практично на нескінченності або у фокальній площині лінзи, що збирає дифраговані промені
(рис 1.). У цьому випадку положення точки спостереження визначається кутом $\phi$ між нормаллю
до ґратки та напрямком променів.

Розподіл інтенсивності в дифракційній картині визначається суперпозицією хвиль, що проходять
у точку спостереження від різних щілин дифракційної ґратки. При цьому амплітуди всіх
інтерферуючих хвиль при заданому куті $\phi$ практично однакові, а фази складають арифметичну
прогресію. Точна теорія дифракційної ґратки враховує як інтерференцію хвиль, що входять від
різних щілин, так і інтерференцію хвиль, що виходять з різних точок однієї щілини. Розрахунки
показують, що інтенсивність світла $I$ , що поширюється під кутом $\phi$ до нормалі, дорівнює:
$$I = A^2 \frac{sin^2 (N\frac{kd sin\phi}{2})}{sin^2 (\frac{kd sin\phi}{2})},$$
де $k -\frac{2\pi}{\lambda}$ - хвильове число. \\
Аналіз виразу показує, що при великому числі щілин світло після ґратки поширюється у
певних напрямках, які визначаються співвідношенням
$$sin(\phi) = m \frac{\lambda}{d}$$
Розміщення головних дифракційних мінімумів визначається умовою:
$$sin(\phi) = m \frac{\lambda}{a}$$
де $a$ - ширина однієї щілини. Ці мінімуми знаходяться на значно більших відстанях, ніж
інтерференційні мінімуми, оскільки постійна $d$ ґратки значно більша за ширину щілини. \\
Між кожними сусідніми головними максимумами виникає $(N-1)$ додаткових інтерференційних
мінімумів. Напрямки пучків світла, в яких виникають мінімуми інтенсивності, визначаються
умовами:
$$d sin(\phi) = \frac{\lambda}{N}, \frac{2\lambda}{N},...,\frac{(N-1)\lambda}{N}$$
Чим більше щілин у ґратці, тим тоншими й інтенсивнішими будуть головні максимуми
Дифракційні ґратки застосовуються для вимірювання довжин хвиль світла. Якщо виміряти кути
$\phi$ , утворені невідхиленими ґраткою променями з напрямками головних максимумів для деякого
монохроматичного світла, то за формулою можна обчислити довжину хвилі цього світла по
відомому періоду ґратки. Визначення напрямків головних максимумів можна провести за
допомогою гоніометра.\\
Основними характеристиками дифракційної ґратки є кутова дисперсія та роздільна здатність.
Кутова дисперсія $D_{\phi}$ визначає кутову віддаль між двома спектральними лініями з різницею
довжин хвиль $\Delta \lambda = 1 nm$.
$$D_{\phi} = \frac{d\phi}{d\lambda}$$
Диференціюючи обидві частини рівняння, одержимо:
$$ d cos(\phi) d \phi = m d \lambda $$
$$D_{\phi} = \frac{d \phi}{d \lambda} = \frac{m}{d cos(\phi)} = \frac{m}{\sqrt{d^2 - m^2 {\lambda}^2}}$$

Роздільна здатність ґратки встановлюється критерієм Релея, згідно з яким дві лінії спектра
однакової інтенсивності з довжинами хвиль $\lambda$ та $(\lambda + \Delta \lambda)$ видні окремо (тобто розділяються),
якщо головний максимум першої хвилі попадає в найближчий інтерференційний мінімум другої
хвилі. Теоретично роздільна здатність $R$ умовно визначається через інтервал довжин хвиль $d \lambda$
між максимумами цих ліній:
$$R = \frac{\lambda}{d \lambda}$$
На основі умови Релея із формул і одержуємо таку рівність:
$$ m( \lambda + d \lambda) = (mN+1)\frac{\lambda}{N}$$
$$d \lambda = \frac{\lambda}{N \cdot m}$$
$$R = \frac{\lambda}{d \lambda} = mN$$
Таким чином, при роботі з дифракційною ґраткою велика роздільна здатність досягається за
рахунок великих значень числа штрихів ґратки $N$ , оскільки порядок m звичайно невисокий.
