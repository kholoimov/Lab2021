\section{Вступна частина}
\setlength{\parindent}{4em}
\indent \textbf{Мета роботи:} засвоїти інтерференційний метод вимірювання довжини хвилі за допомогою
біпризми Френеля. \par
\textbf{Прилади:} оптична лава, джерело світла (ртутна лампа або газовий лазер), конденсор, щілинна діафрагма, світлофільтри, біпризма Френеля, окулярний мікрометр,
теодоліт.
\begin{center}
\textbf{\emph{Теоретичні відомості}}
\end{center}
\indent Головні складові рефрактометра - дві призми $K$ та $K'$. Призми повернуті одна до одної діагональними поверхнями, між якими утворюється тонкий плоскопаралельний прошарк. Грань АВ верхньої призми матова і служить для освітлення розсіяним світлом рідини. Нижня, вимірююча призма $K'$зроблена з важкого флінту $(n>1,7)$ і має
величину заломлюючого кута близьку до $60°$. Призма $K$ зв’язана з шарніром і може бути відведена від призми $K'$. 2-3 краплі досліджуваної рідини капають на гіпотенузну грань вимірювальної призми $K'$ і опускають на неї освітлювальну призму так, що між призмами залишається тонкий шар рідини. \\
Світло, розсіяне матовою поверхнею верхньої призми, проходить тонкий шар досліджуваної рідини і падає на діагональну грань вимірювальної призми під різними кутами в межах від 0 до $90º$. Промінь світла, кут падіння якого дорівнює $90º$, називається ковзним. Оскільки показник заломлення досліджуваної рідини менший за показник заломлення призми $K'$, то ковзний світловий пучок, заломлюючись на межі рідина-скло, піде у нижній призмі під граничним кутом заломлення ${\beta}_{гр}$. Світло, розсіяне матовою поверхнею верхньої призми, проходить тонкий шар досліджуваної рідини і падає на діагональну грань вимірювальної призми під різними кутами в межах від 0 до $90º$. Промінь світла, кут падіння якого дорівнює $90º$, називається ковзним. Оскільки показник заломлення досліджуваної рідини менший за показник заломлення призми
$K'$, то ковзний світловий пучок, заломлюючись на межі рідина-скло, піде у нижній призмі під граничним кутом заломлення. При всіх кутах падіння $\alpha<90º$ світлові промені заломлюються під меншими кутами, виходять з призми $K'$ під більшими кутами до її гіпотенузи і дадуть зображення в зоровій трубі в точках фокальної площини нижче, ніж точка для променів з граничним кутом ${\beta}_{гр}$. Таким чином, нижня половина поля зору в трубі буде світлою, а верхня - темною, оскільки паралельні промені з кутами заломлення, більшими, ніж ${\beta}_{гр}$ відсутні. Межею світла та тіні є промінь з
граничним кутом. \\
При використанні білого світла замість різкої межі між світлою і темною частинами поля зору спостерігається райдужне, кольорове забарвлення межі внаслідок розкладання білого променя. Кольорове забарвлення межі зумовлене залежністю показників заломлення досліджуваної рідини та вимірювальної призми від довжини хвилі світла.
Щоб уникнути кольорового забарвлення межі, в рефрактометрі є спеціальний компенсатор дисперсії, який встановлено перед об’єктивом зорової труби. Компенсатор складається з двох призм прямого зору – призм Амічі, які можуть обертатися навколо оптичної осі зорової труби в протилежних напрямках. Призма Амічі складається з трьох призм: двох крайніх з крону і середньої з флінту, причому вони підібрані так, щоб жовті промені проходили цю систему без відхилення. Можна встановити призми Амічі в такому положенні, що їх сумарна дисперсія в напрямку, перпендикулярному до межі світла і тіні, компенсувала дисперсію досліджуваної рідини і вимірювальної призми. При цьому межа світла-тіні стане незабарвленою і її положення співпаде з положенням межі, яку утворює жовте світло. Тому шкала рефрактометра градуюється
в значеннях показника заломлення $n_D$ для довжини хвилі жовтої лінії натрію. За величиною кута повороту компенсатора можна приблизно визначити величину дисперсії досліджуваної рідини. Для цього на ручку компенсатора нанесена відповідна шкала. Якщо відомі показник заломлення призми, заломлюючий кут призми і кут, під яким
спостерігається межа світла і тіні, можна розрахувати показник заломлення досліджуваної рідини. На цьому принципі заснована робота лабораторного рефрактометра. У рефрактометрі РЛ-2 на шкалі приладу вже нанесені значення показника заломлення рідин, отже необхідно
візирну лінію, яку видно в окуляр зорової труби, на границю світла й тіні, і тоді ми одержимо $n$ вимірюваної речовини.




\newpage
\begin{center}
  {\textbf{\emph{Визначення фокусної відстані та положення головних площин товстої
збірної лінзи.}}}
\end{center}
Фокусну відстань товстої збірної лінзи визначають за способом Аббе (рис. 2). Нехай
предмет $y$ знаходиться на відстані $-X_1$
 від головного фокуса $F$ товстої збірної лінзи.
Зображення предмета має розмір $-y_1'$\\
Лінійне збільшення ${\beta}_1$ буде:
$${\beta}_1 = \frac{y_1'}{y}=-\frac{f}{X_1}$$
Якщо пересунути предмет $y$ в положення ,То лінійне збільшення буде:
$${\beta}_2 = \frac{y_2'}{y}=-\frac{f}{X_2}$$ \\
З цих формул можна отримати вираз для фокусної відстані:
$$f=-f'=\frac{X_2-X_2}{\frac{1}{{\beta}_1}-\frac{1}{{\beta}_2}}=\frac{\Delta y_1' y_2'}{y(y_2'-y_1')}$$
\begin{center}
  {\textbf{\emph{Визначення фокусної відстані тонкої розсіюючої лінзи.}}}
\end{center}
Визначення фокусної відстані розсіюючої лінзи ускладнюється тим, що зображення
дійсних предметів одержуються уявними і не можуть бути безпосередньо виміряні. Це
ускладнення можна усунути двома способами.\\
У першому способі додатково використовується збірна лінза. На початку досліду на
оптичній лаві розміщують лише одну збірну лінзу  і одержують на екрані дійсне
зображення предмета '
$A$ , яке служитиме уявним предметом для розсіюючої лінзи. По лінійці
оптичної лави відмічають його положення. Потім на шляху променів, що виходять із збірної
лінзи, розміщують досліджувану розсіюючу лінзу. Зображення предмета переміститься
тепер у більш віддалену точку
$A''$ . Відмічаючи по лінійці оптичної лави положення $A''$ і координату розсіюючої лінзи $C$ , визначають відстань $A'C$ та $A''C$ і за формулою обчислюють $f'$ розсіюючої лінзи.\\
У другому способі, крім збірної лінзи, використовується ще зорова труба. Якщо
зображення, що дається збірною лінзою, співпадає з переднім фокусом розсіюючої лінзи, то
після заломлення в ній промені вийдуть з лінзи паралельним пучком. Паралельність пучка можна
встановити за допомогою зорової труби, настроєної на нескінченність. Знаючи положення
розсіюючої лінзи, а також положення її головного фокуса, легко визначити фикусну відстань,
якщо лінза тонка. Якщо розсіююча лінза товста, даний метод дозволяє визначити лише
положення фокуса
