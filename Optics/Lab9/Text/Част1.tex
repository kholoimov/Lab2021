\section{Вступна частина}
\setlength{\parindent}{4em}
\begin{center}
\textbf{\emph{Теоретичні відомості}}
\end{center}
\qquad Механізм утворення кілець Ньютона – інтерференція у тонкій плівці повітря, яка утворюється, якщо притиснути лінзу до скляної пластинки. Тоді, за рахунок змінної товщини плівки, власне і будуть спостерігатися кільця – інтерференційні максимуми та мінімуми. Різниця ходу виражається формулою $$\Delta = 2 \delta_{m} + \frac{\lambda}{2}$$\\ тут $\delta_{m} = \frac{{r^2}_{m}}{2R}$ \\
Звідси, а також з умов утворення мінімумів $\Delta = m \lambda + \frac{\lambda}{2}$ одержуємо:
$$r_{m} = \sqrt{mR \lambda}$$ \\
Для максимумів аналогічно
$$r_{m} = \sqrt{(m+\frac{1}{2})R \lambda}$$
