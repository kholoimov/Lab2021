\section{Вступна частина}
\setlength{\parindent}{4em}
\indent \textbf{Мета роботи:} ознайомлення з конструкцією мікроінтерферометра та з інтерференційним
методом контролю якості полірованих поверхонь \par
\textbf{Прилади:}  мікроінтерферометр Лінника, зразок
\begin{center}
\textbf{\emph{Теоретичні відомості}}
\end{center}
\indent Для контролю за чистотою обробки поверхонь високого класу точності, для
вимірювання товщин плівок, тощо В. П. Лінник у 1938 році розробив прилад, який являє собою
комбінацію інтерферометра і мікроскопа. Цей прилад дістав назву мікроінтерферометра.\\
Основну ідею інтерференційного способу контролю якості поверхні можна з
розуміти з рис. 1а, на якому подано принципову оптичну схему інтерферометра Майкельсона.
Розглянемо деякий промінь М, що падає на напівпрозору посріблену пластинку П1, де він
розділяється на два промені 1 і 2. промінь 1 проходить через плоскопаралельну пластинку П2,
роль якої пояснюється нижче, а потім, відбиваючись від дзеркала $Z_1$ та знову від посрібленої
поверхні пластинки $П_1$, поширюється в напрямку спостереження $Н$. Промінь 2 після виходу із
світлороздільної пластинки $П_1$ досягає дзеркала $Z_2$, відбивається і також поширюється в
напрямку спостереження $Н$. Промені 1 і 2 є когерентними, оскільки виникли з одного вихідного
променя $М$. Різниця ходу між ними визначається різницею оптичних довжин плечей
інтерферометра (плечима інтерферометра називаються віддалі від точки поділу променя М на
промені 1 та 2 до дзеркал приладу $Z_1$ та $Z_2$). \\
Промінь 1 проходить через пластинку $П_1$ один раз, а промінь 2- тричі. Легко
переконатись, що внаслідок цього між променями 1 і 2 виникає різниця ходу $\Delta = 2d n cos(\theta)$, де
$d$- товщина пластинки, $n$- показник заломлення пластинки, $\theta$ - кут заломлення світла в пластинці
 Щоб інтерферометр працював в області низьких порядків інтерференції, необхідно
компенсувати вказану різницю ходу. Слід зауважити що ця різниця ходу залежить від довжин
хвилі світла, внаслідок дисперсії показника заломлення матеріалу пластинки $n$. Однозначна
компенсація для всіх довжин хвиль досягається розміщенням на шляху променя 1
плоскопаралельної пластинки $П_2$ такої ж товщини, як $П_1$ і виготовленої з того самого матеріалу.
Цілком зрозуміло, що пластинка $П_2$ необхідна лише в приладах, призначених для роботи в
білому світлі. \\
Простою побудовою зображення одного із дзеркал ($Z_1$ та $Z_2$) в напівпрозорому
дзеркалі світлороздільної пластинки $П_1$ можна переконатися, що це зображення в загальному
випадку утворює кут $\phi$ з другим дзеркалом. \\
Поверхні дзеркал (дійсна та уявна) утворюють повітряний клин. Відомо, що при
освітленні такого клина паралельним пучком світла можна спостерігати так звані смуги рівної
товщини. Звичайно в такому інтерферометрі одне із дзеркал ставлять перпендикулярно до
оптичної осі приладу, а повітряний клин утворюється внаслідок невеликого відхилення другого
дзеркала від перпендикулярного положення. Хід променів поблизу повітряного клина подано на
 з цього малюнка видно, що точка перетину променя 1 з продовженням променя 2 лежить
на поверхні дзеркала $Z_1$. Це означає, що саме в цій площині локалізована інтерференційна
картина. Таким чином, спостерігаючи цю картину смуг рівної товщини можна, розглядаючи
площину дзеркала $Z_1$ безпосередньо оком або з допомоги лупи, мікроскопа, тощо. Смуги рівної
товщини в клині мають форму рівновіддалених прямих ліній, паралельних до ребра клина. При
спостереженні в білому світлі на лінії, по якій перетинаються поверхні $Z_1$ та $Z_2$, спостерігається
незабарвлена смуга нульового порядку ( $\Delta = 0$   ). При контролі якості поверхні замість одного із
дзеркал використовують досліджувану поверхню. В місцях виступів або западин різниця ходу,
обумовлена товщиною клина в деякому місці, змінюється, що призводить до викривлення
інтерференційних смуг в той чи інший бік відносно смуги нульового порядку \\
Величина нерівностей визначається за формулою:
$$t = \frac{\lambda}{2} \cdot \frac{a}{b}$$
де $b$- відстань між інтерференційними смугами, $a$ - величина викривлення
інтерференційних смуг, $\lambda$ - довжина хвилі світла.\\
Оскільки при дослідженні якості обробки поверхонь високого класу точності дефекти
мають дуже малі розміри, то досліджувану поверхню разом з локалізованою на ній
інтерференційною картиною необхідно розглядати з великим збільшенням, наприклад, за
допомогою мікроскопа.
\begin{center}
  {\textbf{\emph{Порядок виконання роботи}}}
\end{center}
\indent Оскільки в даному приладі виміри проводяться в білому світлі, для якого середня
довжина хвилі $\lambda = $ 0,55 мкм, то глибина нерівності
$$t = 0,27 \cdot \frac{a}{b}$$
При роботі в білому світлі всі вимірювання проводяться по двох чорних смугах. Для
більшої точності вимірів слід наводити перехрестя ниток не на край, а на середину смуги.
Величину інтервалу між смугами виражають в поділках барабана гвинтового окулярного
мікрометра. Перший відлік $N_1$ роблять по шкалі мікрометра, коли одна з ниток перехрестя
рухомої сітки наведена на середину смуги. Потім суміщають цю ж нитку з серединою будь-якої
іншої смуги і роблять відлік $N_2$. При цьому необхідно визначити число інтервалів $n$ між смугами.
Величину згину смуг також виражають в поділках барабана гвинтового окулярного мікрометра. \\
Одну з ниток перехрестя суміщають з серединою смуги і поводять відлік $N_3$ по шкалі і
барабану окулярного мікрометра, потім нитку перехрестя суміщають з тією ж смугою в місці
згину і додержують другий відлік $N_4$. Вираз згину смуги в долях інтервалу між смугами має
вигляд:
$$\Delta N = \frac{N_3 - N_4}{N_1-N_2} \cdot n$$
