
\begin{center}
  \textbf{Похибку у роботі обчислюємо наступним чином:}
\end{center}
\begin{enumerate}
  \item Обчислюємо стандартну похибку за формулою $S_x = \sqrt{\frac{\sum\limits_{i=1}^n{x_i - x}^2}{n(n-1)}}$
  \item Обчислюємо випадкову похибку $\Delta x_{вип} = t(\alpha , n) S_x$,$t(\alpha, n )$ - коефіцієнт Стьюдента.
  \item $\Delta x_{instr} = 0,005m$
  \item {$\Delta x = \sqrt{(\Delta x_{вип})^2 + (\Delta x_{instr})^2}$}
  \item $F_{рез} = F_{ser} \pm \Delta F$
\end{enumerate}
\subsection{Отримані результати для фокусної відстані збиральної лінзи:}
Збиральна лінза $F = 0,1 \pm 0,01$ \\
Розсіювальна лінза $F = 0,125 \pm 0,01$\\
\section{Висновок}
\setlength{\parindent}{4em}
\qquad Ми провели есксперименти, у яких різними методами визначили фокусну відстань збиральної та розсіювальної лінзи. Результати вимірів різними методами дали однаковий результат, тому можемо зробити висновок, що досліди проведені правильно.\\
Відносна похибка:\\ $\xi_1 =\frac{0,01}{0,1} = 10\%$ \\
$\xi_2 =\frac{0,01}{0,125} = 8\%$
