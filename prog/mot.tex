\documentclass[a4paper,12pt]{article}

\usepackage{cmap}					% поиск в PDF
\usepackage[T2A]{fontenc}			% кодировка
\usepackage[utf8]{inputenc}			% кодировка исходного текста
\usepackage[english,russian]{babel}	% локализация и переносы
\usepackage[left=2cm,right=2cm,top=2cm,bottom=2cm,bindingoffset=0cm]{geometry}
\usepackage{graphicx}
\usepackage{float}%"Плавающие" картинки
\usepackage{wrapfig}%Обтекание фигур (таблиц, картинок и прочего)
\usepackage{mathtext}
\usepackage{mathtools}
\usepackage{enumitem}
\usepackage{indentfirst}
\setlength{\parskip}{0em}
\setlength{\parindent}{4em}


\author{Холоімов Валерій}
\title{1.1 Наш первый документ}
\date{\today}

\begin{document} % Конец преамбулы, начало текста.
\begin{center}
{\Huge Мотиваційний лист на посаду Голови Студентського парламенту фізичного факультету}
\\
\end{center}
Я, Холоімов Валерій Вячеславович, студент 2 курсу, подаю свою кандидатуру на вибори Голови студентського парламенту фізичного факультету КНУ ім. Т.Шевченка.\\
\hspace*{17mm}Я вивішив балотуватися для того, щоб привнести на факультет усі ті знання, що були отримані мною в результаті плідної роботи в студентських організаціях. Саме студентське самоврядування має відігравати важливу роль у житті студентів. Досвід інших допоможе тобі подолати власні проблеми, підтримка виручить у складних випадках.\\
\begin{center}
{Що ж власне я робив:\\}
\end{center}
{ \textbf{Івенти:}}
\linespread{0.7}
\begin{enumerate}
  \item Як покращити своє CV
  \item Що таке Linkedin та як він працює
  \item Організація івенту за грантом від Національної молодіжної ради України
  \item Робота з бізнесменами, волонтерами, психологами, сексологами
  \item Erasmus дні для студентів КНУ, КПІ та НАУКМА
\end{enumerate}
{ \textbf{Партнерства:}}
\begin{enumerate}
  \item Liki24
  \item Monobank
\end{enumerate}
\linespread{1}
{\hspace*{17mm}Основні проєкти, у створенні яких я брав участь були описані до цього, і ці івенти ми плануємо провести в межах СПФФ. Девізом ESN Kyiv є "Students helping students"  : саме цю ідею я хочу привнести і поширити у стінах факультету. Студентський парламент
має частiше комунiкувати з студентами, задля кращого розумiння їх бажань та потреб.\\}
{\hspace*{17mm}Що робити окрім навчання?\\
{\hspace*{17mm}Вирішити це питання - ще одна мета мого подання на пост кандидата Голови СПФФ.Сьогодні досить важливим є різнобічний розвиток кожного студента. Окрім того, що ти будеш неймовірно розумним фізиком, ти так само маєш знати як подолати психологічні проблеми, як правильно слідкувати за власним здоров'ям та як зробити навколишній світ кращим. Адже хто, якщо не ми? \\









\end{document}
